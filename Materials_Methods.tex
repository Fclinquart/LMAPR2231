\section{Materials and Methods}

The experiments were conducted following the procedures outlined in the reference document \textit{LaboPAC 2025}, available on the Moodle website of the course \cite{labopac2025}, which provides a detailed description of the materials, equipment, and methodologies used.
The laboratory session was divided into two lab sessions. The  first lab session focused on the basic characterization of the water electrolyzer and the PEM fuel cell, while the second lab session involved a more in-depth analysis of the electrochemical processes occurring in fuel cells.
Table. \ref{tab:experimental_summary} is a brief summary of the experimental setup and Objective for the first lab session, which was divided into two main parts: the water electrolyzer experiments and the PEM fuel cell experiments.
Four experiments were conducted during the first lab session. In these document, the experiments are referred to as \textit{Exp1}, \textit{Exp2}, \textit{Exp3}, and \textit{Exp4}.
{\color{red} Preciser quel electerolyseur on a eu et quel fuel cell. + Discuter des resisatnce utilisées}

\begin{table}[H]
    \centering
    \caption{Summary of the four main experimental quantities for Lab Session I, grouped by device type. The analysis done in each experiment is also indicated.}
    \renewcommand{\arraystretch}{2} % Adjust row height
    \setlength{\tabcolsep}{6pt} % Adjust column spacing
    \begin{tabular}{|p{2cm}|p{4cm}|p{4cm}|p{4cm}|}
        \hline
        \textbf{Experiment} & \textbf{Objective} & \textbf{Measured Quantities} & \textbf{Analysis} \\
        \hline
        \multicolumn{4}{|c|}{\textbf{A – Water Electrolyzer Experiments}} \\
        \hline
        \textit{Exp1} & Determine the water decomposition voltage & Voltage (\textbf{U}), Current \textbf{(I)} for various resistances & Plot {current} vs. {voltage}, identify gas onset voltage, compare with theoretical value \\
        \hline
        \textit{Exp2} & Evaluate efficiency of hydrogen production & Time \textbf{t}, Applied voltage \textbf{(U)}, Current \textbf{I}, Volume of ${H_2}$ $\mathbf{v_{H_2}} $produced & Plot $H_2$ volume vs. time, calculate energy and Faraday efficiencies \\
        \hline
        \multicolumn{4}{|c|}{\textbf{B – PEM Fuel Cell Experiments}} \\
        \hline
        \textit{Exp3} & Identify maximum power output of the cell & Voltage \textbf{ (V)}, Current \textbf{ (A)} for various resistances & Plot {voltage} vs. {current} and {power} vs. {current}, determine Maximum Power Point (MPP) \\
        \hline
        \textit{Exp4} & Evaluate efficiency during hydrogen consumption & Time \textbf{t}, Voltage \textbf{U}, Current \textbf{I}, Volume of ${H_2}$ $\mathbf{v_{H_2}}$ consumed & Plot ${H_2}$ consumption vs. time, calculate energy and Faraday efficiencies \\
        \hline
    \end{tabular}
    \vspace{0.5em}
    
    \label{tab:experimental_summary} % Added label for referencing
\end{table}

